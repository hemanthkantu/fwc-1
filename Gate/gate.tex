\documentclass[a4paper ,10pt]{article}
\usepackage[left=1in, right=0.75in, top=1in, bottom=0.75in]{geometry}
\usepackage{graphicx} % Required for inserting images
\usepackage{siunitx}
\usepackage{setspace}
\usepackage{gensymb}
\usepackage{xcolor}
\usepackage{caption}
%\usepackage{subcaption}
\doublespacing
\singlespacing
\usepackage[none]{hyphenat}
\usepackage{amssymb}
\usepackage{relsize}
\usepackage[cmex10]{amsmath}
\usepackage{mathtools}
\usepackage{amsmath}
\usepackage{commath}
\usepackage{amsthm}
\interdisplaylinepenalty=2500
%\savesymbol{iint}
\usepackage{txfonts}
%\restoresymbol{TXF}{iint}
\usepackage{wasysym}
\usepackage{amsthm}
\usepackage{mathrsfs}
\usepackage{txfonts}
\let\vec\mathbf{}
\usepackage{stfloats}
\usepackage{float}
\usepackage{cite}
\usepackage{cases}
\usepackage{subfig}
%\usepackage{xtab}
\usepackage{longtable}
\usepackage{multirow}
%\usepackage{algorithm}
\usepackage{amssymb}
%\usepackage{algpseudocode}
\usepackage{enumitem}
\usepackage{mathtools}
%\usepackage{eenrc}
%\usepackage[framemethod=tikz]{mdframed}
\usepackage{listings}
%\usepackage{listings}
\usepackage[latin1]{inputenc}
%%\usepackage{color}{   
%%\usepackage{lscape}
\usepackage{textcomp}
\usepackage{titling}
\usepackage{hyperref}
%\usepackage{fulbigskip}   
\usepackage{tikz}
\usepackage{graphicx}
\lstset{
  frame=single,
  breaklines=true
}
\let\vec\mathbf{}
\usepackage{enumitem}
\usepackage{graphicx}
\usepackage{siunitx}
\let\vec\mathbf{}
\usepackage{enumitem}
\usepackage{graphicx}
\usepackage{enumitem}
\usepackage{tfrupee}
\usepackage{amsmath}
\usepackage{amssymb}
\usepackage{mwe} % for blindtext and example-image-a in example
\usepackage{wrapfig}
\graphicspath{{figs/}}
\providecommand{\mydet}[1]{\ensuremath{\begin{vmatrix}#1\end{vmatrix}}}
\providecommand{\myvec}[1]{\ensuremath{\begin{bmatrix}#1\end{bmatrix}}}
\providecommand{\cbrak}[1]{\ensuremath{\left\{#1\right\}}}
\providecommand{\brak}[1]{\ensuremath{\left(#1\right)}}
\usetikzlibrary{decorations.pathreplacing,backgrounds}
\usepackage{circuitikz}

\title{GATE-EC2021,31}
\begin{document}
\maketitle
\begin{enumerate}
    \item The propagation delays of the XOR gate, AND gate and multiplexer \brak{MUX} in the circut shown in the figure are $4 ns$, $2 ns$ and $1 ns$, respectively.
    If all the inputs $P, Q, R, S$ and T are applied simultaneously and held constant, the maximum propagation delay of the circuit is
    
    \end{enumerate}

\begin{figure}[H]
\begin{circuitikz}    
\draw (7,1)coordinate (E) -- (8,1)coordinate (F) -- (8,-1)coordinate (G) -- (7,-1)coordinate (H) -- (7,1)coordinate (E);
\draw (11,2)coordinate (I) -- (13,2)coordinate (J) -- (13,-1)coordinate (K) -- (11,-1)coordinate (L) -- (11,2)coordinate (I);
 \draw ($(J)!0.5!(K)$)--++(0:2)node[right]{$Y$};
 \draw ($(L)!0.5!(K)$)node[anchor=south]{$S0$}--++(90:-2)--++(1:-10)node[left]{$T$};
 \draw ($(H)!0.5!(G)$)node[anchor=south]{$S0$}--++(90:-2)--++(1:0)node[left]{};
\draw (4,2) node[and port] (myand1) {};
\draw (myand1.in 1) node (A1)     [anchor=east,xshift=-1cm]           {$P$}
(myand1.in 1) -- (A1);
\draw(4,-2) node[and port] (myand2) {}
(myand2.in 2) node (B2)     [anchor=east,xshift=-1cm]           {$S$};
\draw(myand2.in 2) -- (B2);
\draw (10,0) node[and port] (myand3) {};
\draw (4,0) node[xor port] (myxor) {};
\draw (myand1.in 2) node (B1)     [anchor=east,xshift=-1cm,yshift=-.7cm]  {$Q$};
\draw (B1) -- ++(1.25cm,0);
\draw (myxor.in 1) node (B1)     [anchor=east,xshift=-1cm,yshift=-1.3cm]  {$R$};
\draw (B1) -- ++(1.25cm,0);

\draw (myand1.in 2) |- (myxor.in 1);
\draw (myand2.in 1) |- (myxor.in 2);

\draw (myxor.out) |- ($(E)!0.2!(H)$)--++(0:0)node[right]{$0$};
\draw(myand1.out) |- ($(I)!0.2!(L)$)--++ (0:0)node[right]{$0$};
\draw (myand1.out) -| (myand3.in 1);
\draw(myand3.out) -| ($(I)!0.7!(L)$)--++(0:0)node[right]{$1$};
\draw ($(F)!0.6!(G)$)--++(0:0)node[right]{} -| (myand3.in 2) ;
\draw (myand2.out) |- ($(E)!0.7!(H)$)--++(0:0)node[right]{$1$};


\end{circuitikz}
	\caption{Circuit Diagram}
	\label{fig:block_diagram}
\end{figure}
\begin{enumerate}

    \item $3 ns$
    \item $5 ns$
    \item $6 ns$
    \item $7 ns$
\end{enumerate}

\end{document}
